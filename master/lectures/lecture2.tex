\newpage
\section{September 11, 2023}
\subsection{Examples of metric spaces}

Another metric for the set $\mathbb{R}^n$ is the mapping $d_1$, where
\[
d_1(x, y) = \sum_{k=1}^{n} |x_k - y_k|.
\]
This also reduces to the metric of Example (1) when $n = 1$.

Both the Euclidean metric and the metric $d_1$ just defined are special cases of the metric $d_p$, where
\[
d_p(x, y) = \left( \sum_{k=1}^{n} |x_k - y_k|^p \right)^{\frac{1}{p}},
\]
with $p \geq 1$. The verification of (M3) for this mapping for general values of $p$ requires a discussion of the H\"{o}lder inequality and the Minkowski inequality.



(6) A third metric for the set $\mathbb{R}^n$ is given by the mapping $d_{\infty}$, where
\[
d_{\infty}(x, y) = \max_{1 \leq k \leq n} |x_k - y_k|.
\]
When $n = 1$, we again obtain the metric of Example (1), while when $n = 2$ we obtain that of Example (3). The method of Example (3) is used in showing that $d_{\infty}$ is a metric.

\subsection{Hölder and Minkowski inequalities}
\begin{example}
Let \( X = \mathbb{R}^n \) and \( d_p(x, y) = \left( \sum_{k=1}^{n} |x_k - y_k|^p \right)^{\frac{1}{p}} \) for \( p \geq 1 \).

To prove (M3), we use:

\textbf{Hölder inequality}
\[
\left| \sum_{k=1}^{n} a_k b_k \right| \leq \left( \sum_{k=1}^{n} |a_k|^p \right)^{\frac{1}{p}} \cdot \left( \sum_{k=1}^{n} |b_k|^q \right)^{\frac{1}{q}}
\]
where \( \frac{1}{p} + \frac{1}{q} = 1 \) (extension of Cauchy-Schwarz inequation).

\textbf{Goal}: Prove inequality:
\[
\left( \sum_{k=1}^{n} |a_k + b_k|^p \right)^{\frac{1}{p}} \leq \left( \sum_{k=1}^{n} |a_k|^p \right)^{\frac{1}{p}} + \left( \sum_{k=1}^{n} |b_k|^p \right)^{\frac{1}{p}} \quad (\text{Minkowski})
\]

Then (M3) follows taking \( a_k = x_k - y_k \) and \( b_k = y_k - z_k \).

\textbf{Exercise}: Check this. (Hint: Look at case \( p=2 \), last time)

\end{example}
\begin{proof}
\textbf{Proof of (1)}:
\[
\left| \sum_{k=1}^{n} a_k b_k \right|^p = \left| a_k b_k \right| \left| a_k + b_k \right|^{p-1} \leq \left( |a_k| + |b_k| \right) \left| a_k + b_k \right|^{p-1}
\]
\[
\leq \left( \sum_{k=1}^{n} |a_k|^p \right)^{\frac{1}{p}} \cdot \left( \sum_{k=1}^{n} \left| a_k + b_k \right|^{(p-1)q} \right)^{\frac{1}{q}} \quad \text{by Hölder's inequality with } \frac{1}{p} + \frac{1}{q} = 1
\]
Taking the sum of the previous two inequalities, we get:
\[
\sum_{k=1}^{n} |a_k b_k|^p \leq \left( \sum_{k=1}^{n} |a_k|^p \right)^{\frac{1}{p}} + \left( \sum_{k=1}^{n} |b_k|^p \right)^{\frac{1}{p}}
\]
Also, we know that:
\[
\sum_{k=1}^{n} |a_k + b_k|^p \leq \sum_{k=1}^{n} (|a_k| + |b_k|) |a_k + b_k|^{p-1}
\]
Hence, combining the last two inequalities:
\[
\sum_{k=1}^{n} |a_k + b_k|^p \leq \left( \sum_{k=1}^{n} |a_k + b_k|^p \right)^{\frac{1}{p}} \cdot \left( \sum_{k=1}^{n} \left( |a_k|^p + |b_k|^p \right) \right)^{\frac{1}{p}}
\]
\[
\leq \left( \sum_{k=1}^{n} |a_k|^p \right)^{\frac{1}{p}} + \left( \sum_{k=1}^{n} |b_k|^p \right)^{\frac{1}{p}} \quad \text{by Minkowski's inequality}
\]
\end{proof}

\subsection{$\ell_2$-spaces}
\begin{definition}
    Let $\ell_2$ be the set of sequences $x=\{x_k\}_{k\geq 1}$ with $x_k \in \mathbb{C}$ for all $k \geq 1$ such that $\sum_{k=1}^{\infty} |x_k|^2 < \infty$. We define
\[ d(x,y) = \sqrt{\sum_{k=1}^{\infty} |x_k - y_k|^2} \]
where $x=\{x_k\}_{k\geq 1}$ and $y=\{y_k\}_{k\geq 1}$.
\end{definition}


\begin{lemma}
We justify that it makes sense to define $\sqrt{\sum_{k=1}^{\infty} |x_k - y_k|^2}$ as a metric  by proving:
 \begin{enumerate}
    \item[a)] $d(x,y) < \infty$ for all $x,y \in \ell_2$.
    \item[b)] $d(x,z) \leq d(x,y) + d(y,z)$ for all $x,y,z \in \ell_2$.
\end{enumerate}
\end{lemma}

\begin{proof}
\begin{enumerate}
    \item[a)] If $|x_k - y_k| \leq |x_k| + |y_k|$, then $|x_k - y_k|^2 \leq (|x_k| + |y_k|)^2$ for any $k=1,\ldots,n$, for any $n \geq 1$ fixed. We take the sum for all $k=1,\ldots,n$ and then square root.
    \[ \sqrt{\sum_{k=1}^{n} |x_k - y_k|^2} \leq \sqrt{\sum_{k=1}^{n} (|x_k| + |y_k|)^2} \]
    \[ \leq \sqrt{\sum_{k=1}^{n} |x_k|^2} + \sqrt{\sum_{k=1}^{n} |y_k|^2} \]
    for any $n \geq 1$ by Minkowski's inequality (p=2).
    
    Hence $\sum_{k=1}^{n} |x_k - y_k|^2 \leq M^2 < \infty$ for all $n\geq 1$, and $\sum_{k=1}^{\infty} |x_k - y_k|^2 \leq M^2$.
    
    So $d(x,y) = \sqrt{\sum_{k=1}^{\infty} |x_k - y_k|^2} \leq M < \infty$.

    \item[b)] $|x_k - z_k| = |x_k - y_k + y_k - z_k|$. Take the square $|x_k - z_k|^2 \leq (|x_k - y_k| + |y_k - z_k|)^2$. Take the sum for $k=1,\ldots,n$, then the square root. We get:
    \[ \sqrt{\sum_{k=1}^{n} |x_k - z_k|^2} \leq \sqrt{\sum_{k=1}^{n} (|x_k - y_k| + |y_k - z_k|)^2} \]
    \[ \leq \sqrt{\sum_{k=1}^{n} |x_k - y_k|^2} + \sqrt{\sum_{k=1}^{n} |y_k - z_k|^2} \]
    for any $n \geq 1$ by Minkowski's inequality (p=2).
    
    Then, taking $n \to \infty$ we get:
    \[ d(x,z) \leq d(x,y) + d(y,z) \]
\end{enumerate}
\end{proof}

\begin{remark}
Let $\ell_p$ be the set of all sequences $x=\{x_k\}_{k\geq 1}$ with $x_k \in \mathbb{C}$ such that $\sum_{k=1}^{\infty} |x_k|^p < \infty$. If $p \geq 1$, we define
\[ d(x,y) = \left( \sum_{k=1}^{\infty} |x_k - y_k|^p \right)^{\frac{1}{p}} \]
where $x=\{x_k\}_{k\geq 1}$ and $y=\{y_k\}_{k\geq 1}$. Then $d$ is a distance on $\ell_p$ (Exercise).
\end{remark}

\begin{remark}
If $p \in (0,1)$, we define
\[ d(x,y) = \sum_{k=1}^{\infty} |x_k - y_k|^p \]
This is also a distance. (M3) is verified using another inequality:
\[ |a+b|^p \leq |a|^p + |b|^p \]
for any $a,b \in \mathbb{R}$ (sub-additivity) if $p \in (0,1)$.

Replacement for Minkowski inequality is:
\[ \sum_{k=1}^n |a_k+b_k|^p \leq \sum_{k=1}^n |a_k|^p + \sum_{k=1}^n |b_k|^p \]
\end{remark}

\begin{remark}
Let $\ell_\infty$ be the space of sequences $x=\{x_k\}_{k\geq 2}$, with $x_k \in \mathbb{C}$, such that $\{x_k\}$ is bounded, i.e., $\exists M>0$ such that $|x_k| \leq M$ for all $k\geq 1$.
We define
\[ d(x,y) = \sup_{k \geq 1} |x_k - y_k| \]
Exercise: check that $d$ is a distance on $\ell_\infty$.
\end{remark}

\begin{example}
Let $X = \mathbb{C}$. Consider the distance function $d(x,y) = \frac{|x-y|}{\sqrt{(1+|x|^2)(1+|y|^2)}}$, which is called the chordal distance.
\end{example}
\begin{proof}
The proof that $d$ is a distance is omitted.
\end{proof} 

\begin{example}
Let $X = C[a,b]$ where $C[a,b] = \{ f : [a,b] \rightarrow \mathbb{R} \mid f \text{ is continuous} \}$. Recall (MAT2125) that a function $f: [a,b] \rightarrow \mathbb{R}$ is continuous at $x_0 \in [a,b]$ if for every $\varepsilon > 0$ there exists $\delta > 0$ such that for all $x \in (x_0 - \delta, x_0 + \delta)$ we have $|f(x) - f(x_0)| < \varepsilon$, i.e., $|x-x_0| < \delta$. $f$ is continuous on $[a,b]$ if $f$ is continuous at every $x \in [a,b]$.
\\
\textbf{Observation:} If $f$ is continuous at $x_0$ then $\lim_{x \to x_0} f(x) = f(x_0)$.
\\
Let $d(f,g) = \max_{t \in [a,b]} |f(t) - g(t)|$. Note: $|x-y|$ is a continuous function on $[a,b]$, and $[a,b]$ is compact, hence the maximum is attained, and $d$ is well-defined. $d$ is called the uniform metric.
\end{example}
