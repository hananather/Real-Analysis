\newpage
\section{November 20, 2023}
\subsection{Finite Dimensional Subspaces}
Let \( V \) be a vector space of dimension \( n \) over a field, and let \( \{v_1, \ldots, v_n\} \) be a basis for \( V \). For any vector \( x \in V \), represented uniquely as \( x = \sum_{k=1}^{n} \alpha_k v_k \), we define the \(\infty\)-norm (or maximum norm) of \( x \) as 
\[ ||x||_{\infty} = \max_{1 \leq k \leq n} |\alpha_k|, \]
where \( \alpha_k \) are the coefficients of \( x \) in the basis \( \{v_1, \ldots, v_n\} \).

\begin{lemma}
    If $||x_m - x|| \to 0 \iff \alpha_{mk} \to \alpha_k$
    for all $k=1, ..,n$. Here $x_m = \sum_{k=1}^{n}\alpha_{mk}v_k$ and $x = \sum_{k=1}^{n}\alpha_k v_k$
\end{lemma}


\begin{lemma}
    Let \( V \) be a vector space of dimension \( n \) over a field, with a basis \( \{v_1, \ldots, v_n\} \). Consider the sequences \( \{x_m\} \) and \( \{x\} \) in \( V \), where \( x_m = \sum_{k=1}^{n}\alpha_{mk}v_k \) and \( x = \sum_{k=1}^{n}\alpha_k v_k \). Then, under the \(\infty\)-norm \( ||\cdot||_{\infty} \), the sequence \( \{x_m\} \) converges to \( x \) (i.e., \( ||x_m - x||_{\infty} \to 0 \)) if and only if for each \( k \) from 1 to \( n \), the sequence of coefficients \( \{\alpha_{mk}\} \) converges to \( \alpha_k \) (i.e., \( \alpha_{mk} \to \alpha_k \)).
    \footnote{This lemma states that in a finite-dimensional vector space with a specific basis, for a sequence of vectors to converge to a given vector under the \(\infty\)-norm, it is necessary and sufficient that the sequence of each individual coefficient (associated with the basis vectors) of these vectors converges to the corresponding coefficient of the limit vector.}
\end{lemma}
This convergence behavior is directly related to the nature of the \(\infty\)-norm, which focuses on the maximum absolute value among the coefficients of the vector representation.

\begin{theorem}
    Let \( V \) be a vector space of dimension \( n \). Consider the set
    \[
    Q = \{ x \in V \mid ||x||_{\infty} \leq 1 \}.
    \]
    Then, \( Q \) is a compact set in \( V \) (when \( V \) is equipped with the \(\infty\)-norm).
\end{theorem}
\begin{proof}
    \term{Not required for final} \\
We assume \( V \) is a complex vector space (the case of a real vector space is almost identical). We prove the result by induction on the dimension \( n \) of \( V \). First suppose that \( n = 1 \) and that \( \{v\} \) is a basis of \( V \). Let
\[ Z = \{a \in \mathbb{C} \mid |a| \leq 1\} = B_{\mathbb{C}}(0,1), \]
which we know is compact in \( \mathbb{C} \) (it is a closed and bounded subset of \( \mathbb{C} \), which is the same as \( \mathbb{R}^2 \) topologically). Define
\[ A: Z \to V, \quad Av = av. \]
Then \( A(Z) = B_V(0,1) \). Since the continuous image of a compact set is compact (Theorem 4.4.7), it suffices to show that \( A \) is continuous. Suppose \( \{a_m\} \) is a sequence in \( Z \) converging to \( a \in Z \). Then
\[ \lVert Aa_m - Aa \rVert_{\infty} = \lVert (a_m - a)v \rVert_{\infty} = |a_m - a| \to 0. \]
Hence \( Aa_m \to Aa \) and so \( A \) is continuous.

Now assume the proposition is true for \( n < h - 1 \) for some \( h > 1 \). We want to show that it is true for \( n = h \). Let
\[ B_i = B_{V}(0,1) \]
when \( n = i \), \( i \in \mathbb{N}_+ \). So our inductive assumption is that \( B_{n} \) is compact for \( n < h - 1 \) and we wish to show that \( B_h \) is compact.

Let \( \{x_m\} \) be a sequence in \( B_h \) and let
\[ x_m = \sum_{j=1}^{h} \alpha_{mj}v_j \]
for \( m \in \mathbb{N}_+ \). The sequence \( \{\alpha_{m1}v_1\}_{m=1}^{\infty} \) is a sequence in \( B_1 \), which is compact. Therefore, it has a convergent subsequence \( \{\alpha_{mk_1}v_1\}_{k=1}^{\infty} \). So
\[ \left\{ \sum_{j=2}^{h} \alpha_{mk,j}v_j \right\}_{k=1}^{\infty} \]
is a subsequence of \( \{x_m\} \) such that the sequence of coefficients of \( v_1 \) converges. Then
\[ \left\{ \sum_{j=2}^{h} \alpha_{mk,j}v_j \right\}_{k=1}^{\infty} \tag{5.1} \]
is a sequence in \( \text{Span}\{v_2, \ldots, v_h\} \), which is a vector space of dimension \( h - 1 \). Since, for \( k \in \mathbb{N}_+ \),
\[ \left\lVert \sum_{j=2}^{h} \alpha_{mk,j}v_j \right\rVert_{\infty} = \max_{2 \leq j \leq h} |\alpha_{mk,j}| \leq \max_{1 \leq j \leq h} |\alpha_{mk,j}| = \lVert x_{mk} \rVert_{\infty} \leq 1, \]
it is a sequence in \( B_{h-1} \). Since \( B_{h-1} \) is compact by the inductive hypothesis, the sequence (5.1) has a convergent subsequence. By Theorem 5.3.2, this subsequence converges componentwise. Therefore the corresponding subsequence of \( \{x_m\} \) also converges componentwise (since we had already chosen a subsequence in which the first component converged) and hence converges.

\end{proof}

\begin{definition}[Equivalent Norms]
    Two norms \( || \cdot ||_1 \) and \( || \cdot ||_2 \) on a vector space \( V \) are said to be \textit{equivalent} if there exist constants \( a, b > 0 \) such that for all \( x \in V \),
    \[
    a ||x||_1 \leq ||x||_2 \leq b ||x||_1.
    \]
    This means that the two norms are equivalent if they induce the same topology on \( V \), i.e., a sequence converges in one norm if and only if it converges in the other.
\end{definition}

\begin{theorem}[Equivalence of Norms in Finite-Dimensional Spaces]
    In any finite-dimensional vector space \( V \), any two norms \( ||\cdot||_a \) and \( ||\cdot||_b \) are equivalent. That is, there exist positive constants \( c \) and \( C \) such that for all vectors \( x \in V \),
    \[
    c ||x||_a \leq ||x||_b \leq C ||x||_a.
    \]\footnote{
    This implies that all norms on a finite-dimensional vector space induce the same topological structure, meaning that notions of convergence, continuity, and compactness are the same under any norm.}
\end{theorem}
\begin{proof}
We will only prove that any norm \( \|\cdot\| \) for our vector space \( V \) is equivalent to the norm \( \|\cdot\|_{\infty} \). That is, we will show that there exist positive numbers \( a \) and \( b \) such that
\[
a\|x\|_{\infty} \leq \|x\| \leq b\|x\|_{\infty}
\]
for any \( x \in V \). This readily implies the theorem, but the details are left as an exercise.

In Theorem 6.5.1, we showed that the subset \( Q = \{x : \|x\|_{\infty} \leq 1\} \) of \( V \) is compact. It is another simple exercise to use this fact, in conjunction with Exercise 4.5(3), to conclude that the set \( Q' = \{x : \|x\|_{\infty} = 1\} \) in \( V \) is also compact. On any normed space, the norm is a continuous mapping (Exercise 6.4(3)(c)) so we may invoke Theorem 4.3.2 to ensure the existence of points \( x_M \) and \( x_m \) in \( Q' \) such that
\[
\|x_M\| = \max_{x \in Q'} \|x\|, \quad \|x_m\| = \min_{x \in Q'} \|x\|.
\]
Thus \( \|x_m\| \leq \|x\| \leq \|x_M\| \) for all \( x \in Q' \). Also, since \( \|x_m\|_{\infty} = 1 \), we cannot have \( x_m = 0 \), so \( \|x_m\| > 0 \). For any nonzero vector \( x \in V \), we have
\[
\frac{1}{\|x\|_{\infty}} x = \frac{1}{\|x\|_{\infty}} x \in Q'.
\]
Hence, for \( x \neq 0 \),
\[
\|x_m\| \leq \left\|\frac{1}{\|x\|_{\infty}} x\right\| \leq \|x_M\|.
\]
We thus have
\[
\|x_m\| \|x\|_{\infty} \leq \|x\| \leq \|x_M\| \|x\|_{\infty},
\]
or
\[
a\|x\|_{\infty} \leq \|x\| \leq b\|x\|_{\infty},
\]
where \( a = \|x_m\| > 0 \) and \( b = \|x_M\| \), and this is clearly true also when \( x = 0 \). Hence the norms \( \|\cdot\| \) and \( \|\cdot\|_{\infty} \) are equivalent.
\end{proof}


\begin{theorem}[Finite-Dimensional Spaces are Banach Spaces]
    Every finite-dimensional normed vector space is a Banach space. That is, if \( V \) is a vector space of finite dimension \( n \) equipped with any norm \( ||\cdot|| \), then \( V \) is complete with respect to this norm. In other words, every Cauchy sequence in \( V \) converges to an element in \( V \).
\end{theorem}

\begin{proof}
Let \( V \) be a finite-dimensional vector space with basis \( \{v_1, \ldots, v_n\} \). By Theorem 6.5.3, all norms are equivalent. So, it is enough to prove that \( (V, \|\cdot\|_{\infty}) \) is a Banach space. Let \( \{x_m\} \) be a Cauchy sequence in \( V \) with respect to \( \|\cdot\|_{\infty} \). Then
\[ 
x_m = \sum_{k=1}^{n} \alpha_{mk} v_k \text{ with } \alpha_{mk} \text{ scalars}.
\]
For any \( \varepsilon > 0 \), there exists \( N \in \mathbb{N} \) such that
\[
\|x_m - x_j\|_{\infty} < \varepsilon \text{ for all } m,j > N.
\]
Note that \( x_m - x_j = \sum_{k=1}^{n} \alpha_{mk} v_k - \sum_{k=1}^{n} \alpha_{jk} v_k = \sum_{k=1}^{n} (\alpha_{mk} - \alpha_{jk}) v_k \), and hence
\[
\|x_m - x_j\|_{\infty} = \max_{1 \leq k \leq n} |\alpha_{mk} - \alpha_{jk}| < \varepsilon \text{ for all } m,j > N.
\]
This means that for all \( k=1, \ldots, n \) fixed,
\[
|\alpha_{mk} - \alpha_{jk}| < \varepsilon \text{ for all } m,j > N.
\]
Hence, \( \{\alpha_{mk}\}_{m \in \mathbb{N}} \) is a Cauchy sequence in \( \mathbb{F} \) (where \( \mathbb{F} = \mathbb{R} \) or \( \mathbb{F} = \mathbb{C} \)). Since \( \mathbb{F} \) is complete, there exists \( \alpha_k \in \mathbb{F} \) such that
\[
\alpha_{mk} \rightarrow \alpha_k \text{ as } m \rightarrow \infty, \text{ for } k=1, \ldots, n.
\]
By Lemma 1, we have that \( x_m \rightarrow x \) in \( (V, \|\cdot\|_{\infty}) \) where \( x = \sum_{k=1}^{n} \alpha_k v_k \).
Hence \( \{x_m\} \) converges in \( (V, \|\cdot\|_{\infty}) \). This proves that \( (V, \|\cdot\|_{\infty}) \) is a complete space.
\end{proof}
This theorem is significant because it ensures that many of the convenient properties of finite-dimensional spaces (like
$\RR^n$) hold more generally in any space with a finite number of dimensions, regardless of the specific norm used. It's a fundamental difference between finite-dimensional and infinite-dimensional spaces, where completeness is not guaranteed and depends heavily on the chosen norm.


Next theorem is a key result in the context of finite-dimensional normed vector spaces. It connects the concepts of compactness, sequential closedness, and boundedness.
\begin{theorem}
    Let \( S \) be a subset of a finite-dimensional normed vector space \( V \). Then \( S \) is compact if and only if it is sequentially closed and bounded.
\end{theorem}

This theorem is a variation of the Heine-Borel theorem, adapted to the language of sequential closedness. In finite-dimensional spaces, the concepts of closed and bounded sets and sequentially closed sets are often interchangeable in terms of leading to compactness, which is one of the fundamental distinctions between finite and infinite-dimensional spaces.

\subsection{Approximation Theory}

Let \( (X, d) \) be a metric space, and let \( S \subseteq X \) be a compact subset. For any point \( x \in X \), there exists a point \( p \in S \) such that 
    \[
    d(p, x) = \min_{y \in S} d(y, x),
    \]
which means \( p \) is the closest point in \( S \) to \( x \), or the best approximation of \( x \) in \( S \).



\begin{theorem}
    Let \( (X, ||\cdot||) \) be a normed vector space, and let \( S \subseteq X \) be a finite-dimensional subspace of \( X \). For a fixed point \( x \in X \), and assuming \( S \neq \emptyset \), there exists a point \( p \in S \) such that 
    \[
    ||p - x|| = \min_{y \in S} ||y - x||,
    \]
    which means \( p \) is the closest point in \( S \) to \( x \), and is called the best approximation of \( x \) in \( S \).
\end{theorem}

\begin{proposition}
    Let \( f \) be a function in \( C[0,1] \), the space of continuous functions on the interval \([0,1]\), and let \( r \) be an integer greater than 1. Then, there exists a polynomial \( p \) of degree less than \( r \) such that
    \[
    ||p - f|| = \min_{g \in S_r} ||g - f||,
    \]
    where \( ||f|| = \max_{t \in [0,1]} |f(t)| \) and \( S_r \) is the set of all polynomials of degree less than \( r \).
\end{proposition}

The theorem essentially states that for any continuous function on 
$[0,1]$, there exists a polynomial of degree less than
$r$ that is the best approximation of 
$f$ in the uniform norm, among all polynomials of degree less than $r$. This is a fundamental result in approximation theory and is related to the Weierstrass approximation theorem.