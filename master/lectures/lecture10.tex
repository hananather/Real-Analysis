\newpage
\section{October 16, 2023}
\subsection{Application to approximation theory continued}
\begin{theorem}
    Given a nonempty compact subset $S$ of a metric space $(X,d)$
    and a point $x \in X$, there exists a point $p \in S$ such that $d(p,x)$ is a minimum.\footnote{The point $p$ is called a \textit{best approximation} in $S$ of the point $x$ in $X$}
\end{theorem}

\begin{example}[Application of Ascoli’s theorem]
    Consider a family \( F \) of functions \( f \) defined as
    \[
    f(x) = a \sin bx + c \cos dx, \quad 0 \leq x \leq \pi,
    \]
    where the coefficients \( a, b, c, \) and \( d \) are chosen from a closed interval \([-M, M]\).
    
    First, observe that the functions in \( F \) are uniformly bounded. For any \( f \) in \( F \) and any \( x \) in \([0, \pi]\), we have:
    \[
    |f(x)| \leq |a| + |c| \leq 2M.
    \]
    Additionally, the derivatives of functions in \( F \) satisfy:
    \[
    |f'(x)| \leq |ab| + |cd| \leq 2M^2.
    \]
    Thus, \( F \) is equicontinuous.
    
    Given this, the family \( F \) can be treated as a subset of the continuous functions on the interval \([0, \pi]\). Due to its equicontinuity and boundedness, \( F \) is compact in this function space.
    
    As a result, for any continuous function \( g \) defined on \([0, \pi]\), we can find a function within \( F \) that comes closest to \( g \) in the sense of minimizing the maximum deviation over the interval. Formally, there exist values of \( a, b, c, \) and \( d \) in \([-M, M]\) such that:
    \[
    \max_{0 \leq x \leq \pi} |g(x) - (a \sin bx + c \cos dx)|
    \]
    is minimized. A function in \( F \) achieving this minimal deviation is called a \textit{minimax approximation} to \( g \). It's important to note that this approximating function might not be unique.

\end{example}

\subsection{Topological Spaces}
\begin{definition}[Topological Spaces]
    A \textit{topology} on a nonempty set \( X \) is a collection \( \mathcal{T} \) of subsets of \( X \) satisfying the following properties:

    \begin{enumerate}
        \item \( X \in \mathcal{T} \) and \( \emptyset \in \mathcal{T} \) (Property T1).
        
        \item The union of any subcollection \( \mathcal{S} \) of \( \mathcal{T} \) belongs to \( \mathcal{T} \). Formally, for any subcollection \( \mathcal{S} \) of \( \mathcal{T} \):
        \[
        \bigcup_{T \in \mathcal{S}} T \in \mathcal{T} \quad (\text{Property T2}).
        \]
        
        \item The intersection of any two sets \( T_1 \) and \( T_2 \) in \( \mathcal{T} \) belongs to \( \mathcal{T} \). Formally:
        \[
        T_1 \cap T_2 \in \mathcal{T} \quad \text{whenever} \quad T_1, T_2 \in \mathcal{T} \quad (\text{Property T3}).
        \]
    \end{enumerate}

    The pair \( (X, \mathcal{T}) \) is called a \textit{topological space}. The sets \( T \) in \( \mathcal{T} \) are referred to as the \textit{open sets} in \( (X,\mathcal{T}) \). A subset \( S \) of \( X \) is said to be \textit{closed} in \( (X,\mathcal{T}) \) if its complement, denoted by \( X \setminus S \) or \( \sim S \), is an open set in \( (X,\mathcal{T}) \).

\end{definition}


\begin{example}[Standard Topology on \(\mathbb{R}\)]
    The most common topology on the real numbers is the \textit{standard topology}, which is generated by the open intervals. In this topology, a set \( U \subseteq \mathbb{R} \) is open if for every point \( x \in U \), there exists an open interval \( (a, b) \) such that \( x \in (a, b) \subseteq U \).
\end{example}
\begin{example}[Lower Limit Topology on \(\mathbb{R}\)]
    Another example is the \textit{lower limit topology} on \( \mathbb{R} \), which is generated by the half-open intervals of the form \([a, b)\). A set is open in this topology if it can be expressed as a union of such half-open intervals.
\end{example}

\begin{example}[Why Infinite Intersections Aren't Always Open]
    While the arbitrary union of open sets is open (by definition of a topology), the same isn't true for infinite intersections. To see why, consider the standard topology on \( \mathbb{R} \). Take the nested sequence of open intervals:
    
    \[
    I_n = \left( -\frac{1}{n}, \frac{1}{n} \right), \quad n = 1, 2, 3, \ldots
    \]
    The intersection of all these intervals is \(\{0\}\), which is not an open set in the standard topology on \( \mathbb{R} \). Hence, while each \( I_n \) is open, their infinite intersection is not. This demonstrates the necessity of the limitation in the definition of a topology that only finite intersections of open sets are guaranteed to be open.
\end{example}
\note{
    Given any set \( X \), there are two fundamental topologies:
    The \textit{discrete topology} on \( X \), denoted \( T_{\text{max}} \), is the collection of all subsets of \( X \). Formally,
    \[ 
    T_{\text{max}} = \mathcal{P}(X),
    \]
    where \( \mathcal{P}(X) \) represents the power set of \( X \). In this topology, every subset of \( X \) is considered open.
    The \textit{indiscrete topology} or \textit{trivial topology} on \( X \), denoted \( T_{\text{min}} \), consists of only the empty set and \( X \) itself. Formally,
    \[ 
    T_{\text{min}} = \{\emptyset, X\}.
    \]
    In this topology, no set other than the entire set and the empty set is considered open.
    Clearly, \( T_{\text{max}} \) and \( T_{\text{min}} \) satisfy the properties of a topology, and as their names suggest, they represent the largest and smallest collections of subsets of \( X \) that can be considered as topologies.}



\begin{definition}[Weaker and Stronger Topologies]
    Given two topologies \( \mathcal{T}_1 \) and \( \mathcal{T}_2 \) on a set \( X \), if 
    \[ 
    \mathcal{T}_1 \subseteq \mathcal{T}_2,
    \]
    then \( \mathcal{T}_1 \) is said to be \textit{weaker} (or \textit{coarser}) than \( \mathcal{T}_2 \), and \( \mathcal{T}_2 \) is said to be \textit{stronger} (or \textit{finer}) than \( \mathcal{T}_1 \).
    Given any topology \( \mathcal{T} \) on \( X \), it must hold that 
    \[ 
    \mathcal{T}_{\text{min}} \subseteq \mathcal{T} \subseteq \mathcal{T}_{\text{max}},
    \]
    where \( \mathcal{T}_{\text{min}} \) is the \textit{indiscrete topology} and \( \mathcal{T}_{\text{max}} \) is the \textit{discrete topology}. Thus, among all possible topologies on a set \( X \), the indiscrete topology is always the weakest (or coarsest), and the discrete topology is always the strongest (or finest).
\end{definition}


\begin{example}[Illustrative Example of Topologies on a Finite Set]
    Consider the set \( X = \{1,2,3,4,5\} \) and the collections:
    \begin{align*}
    \mathcal{T}_1 &= \{\emptyset, \{1\}, \{2\}, \{1,2\}, X\}, \\
    \mathcal{T}_2 &= \{\emptyset, \{1\}, \{2\}, \{1,2\}, \{1,2,3\}, \{1,2,3,4\}, X\}, \\
    \mathcal{T}_3 &= \{\emptyset, \{1\}, \{1,2\}, \{1,2,3\}, X\}, \\
    \mathcal{T}_4 &= \{\emptyset, \{1\}, \{2\}, \{1,2\}, \{2,3,4\}, X\}.
    \end{align*}
    It is evident that \( \mathcal{T}_1 \) and \( \mathcal{T}_2 \) are topologies on \( X \). \( \mathcal{T}_1 \) is weaker than \( \mathcal{T}_2 \) since \( \mathcal{T}_1 \subseteq \mathcal{T}_2 \). On the other hand, \( \mathcal{T}_3 \) is another topology for \( X \) that is weaker than \( \mathcal{T}_2 \) but is neither weaker nor stronger than \( \mathcal{T}_1 \).
    For the topological space \( (X, \mathcal{T}_2) \), the closed sets are:
    \[ X, \{2,3,4,5\}, \{1,3,4,5\}, \{3,4,5\}, \{4,5\}, \{5\} \text{ and } \emptyset. \]
    The set \( \{2, 3\} \) is neither open nor closed in this topology. The interior of \( \{2, 3\} \) is \( \{2\} \) and its closure is \( \{2, 3, 4, 5\} \).
    However, \( \mathcal{T}_4 \) is not a topology on \( X \) because the union \( \{1\} \cup \{2,3,4\} = \{1,2,3,4\} \) is not an element of \( \mathcal{T}_4 \), violating the property that arbitrary unions of open sets should remain open.
\end{example}

\begin{definition}[Interior and Closure in Topological Spaces]
    Let \( X \) be a topological space. 
    \item \textbf{Interior:} The \textit{interior} of a subset \( S \) of \( X \) is the union of all open sets contained in \( S \). It is denoted by \(\text{int}(S)\) or \(S^\circ\).
    
    \item \textbf{Closure:} The \textit{closure} of a subset \( S \) of \( X \) is the intersection of all closed sets containing \( S \). It is denoted by \(\text{cl}(S)\) or \(\overline{S}\).
\end{definition}


Every metric space has an associated topology, called the \textit{metric topology}, derived from its metric. This enables us to study metric spaces in the context of topological spaces.
\begin{definition}[Metric Topology in Metric Spaces]
    Let \( (X, d) \) be a metric space.
    \begin{itemize}
    \item \textbf{Open Ball:} For a point \( x_0 \) in \( X \) and a positive real number \( r \), the set
    \[
    \{ x : x \in X, d(x, x_0) < r \}
    \]
    is called an \textit{open ball} in \( X \). It represents the set of all points in \( X \) that are less than \( r \) distance away from \( x_0 \). This set is denoted by \( b(x_0, r) \) and is referred to as the open ball with center \( x_0 \) and radius \( r \).
    
    \item \textbf{Open Sets:} A subset \( T \) of \( X \) is called \textit{open} if \( T = \emptyset \) or for every point in \( T \), there exists an open ball centered at that point which is completely contained in \( T \).
    
    \item \textbf{Metric Topology:} The metric topology \( \mathcal{T}_d \) for \( X \) is the collection of all open sets, as defined above. Thus, a subset \( T \) of \( X \) is in \( \mathcal{T}_d \) if and only if for each \( x \) in \( T \), there exists some radius \( r \) such that the open ball \( b(x, r) \) is a subset of \( T \).
    \end{itemize}
    \footnote{It can be shown that the collection \( \mathcal{T}_d \) of open sets satisfies the properties of a topology, making \( (X, \mathcal{T}_d) \) a topological space. By convention, when we refer to a metric space as a topological space, it is implied that we are considering its associated metric topology.}
\end{definition}



    
    


