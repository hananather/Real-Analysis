\newpage
\section{September 18, 2023}
\begin{definition}[Cauchy sequence]
    A sequence \( \{ x_n \} \) in a metric space \( (X, d) \) is called a \textit{Cauchy sequence} if for every \( \epsilon > 0 \), there exists a positive integer \( N \) such that
    \[
    d(x_n, x_m) < \epsilon \quad \text{whenever} \quad m, n > N.
    \]
    
\end{definition}

\begin{theorem}[Cauchy Convergence Criterion]
    For $X = \RR$, a sequence \( \{ a_n \} \) is convergent if and only if, for every \( \epsilon > 0 \), there exists a positive integer \( N \) such that
    \[
    | a_n - a_m | < \epsilon \quad \text{whenever} \quad m, n > N.
    \]
\end{theorem}
\begin{proof}
    Prove this!
\end{proof}
\begin{definition}[Complete Metric Space]
    If every Cauchy sequence converges in a metric space, then the space is said to be complete 
\end{definition}

\begin{theorem}
    If a sequence in a metric space is convergent, then it is a Cauchy sequence \footnote{The converse is not true!}.
\end{theorem}
\begin{proof}
    Prove this
\end{proof}

\subsection{Examples on completeness}
