\newpage
\section{October 2, 2023}
Before going into the main content we need to re-familiarize ourselves with a basic facts from previous classes (1) The subsequences of convergent real-valued sequences are themselves convergent, and themselves have the same limit as the original sequence. Now the converse is certainly not true. The example 
$\frac{1}{2}, 2, \frac{1}{3}, 3, \frac{1}{4}, 4, \dots $ shows us that a sequence having a convergent subsequence certainly need not itself be converge.

However, we can saw the following:
\begin{theorem}
    In any metric space, a Cauchy sequence having a convergent subsequence is itself convergent, with the same limit. 
\end{theorem}
\begin{proof}
    Let $\{ x_n\}$ be a Cauchy sequence in metric space $(X,d)$, and let $\{ x_{n_k} \}$ be a convergent subsequence of $\{ x_n\}$.
    Set $ x = \lim_{k \to \infty} x_{n_k}$ be the limit of the convergent subsequence. 
    Now for any $\eps > 0$, there exists a $K \in \NN$ such that $d(x_{n_k}, x) < \frac{1}{2}\eps$ when $k > K$. 
    As  $\{ x_n\}$ is a Cauchy sequence, there exists $N \in \NN$ such that for all $n,m > N$, $d(x_n, x_m) < \frac{1}{2} \eps$ when $n,m > N$. We may assume that $K >N$. Which implies that $n_k \geq k > K > N$ and we have
    $$
    d(x_n, x) \leq d(x_n, x_{n_k}) + d(x_{n_k}, x) = \frac{1}{2} \eps + \frac{1}{2} \eps = \eps
    $$
\end{proof}
Before we introduced completeness because of the need to categorize those metric spaces which have the propriety of Cauchy convergence for real numbers. In a similar way we now discuss an other property: \textit{compactness}. The Bolzano–Weierstrass theorem says that there exists a convergent subsequence of any (real-valued) sequence, as long as it is bounded. This notions leads us to our definition of compactness. 
\begin{definition}
    A subset of a metric space is a called \vocab{sequentially compact} if every sequence in the subset has a convergent subsequence.  
\end{definition}

\begin{theorem}
    If a metric space is compact, then it is complete.
\end{theorem}
\begin{proof}
    This follows directly from the previous theorem. A complete metric space is one in which every Cauchy sequence is convergent. From the previous theorem we know if Cauchy sequence has a convergent subsequence, it is convergent. And in the definition of compactness is every sequence in the subset has a convergent subsequence. Therefore, every Cauchy sequence would be convergent in this space.
\end{proof}

\begin{theorem}
    Every compact set in a metric space is bounded. 
\end{theorem}
\begin{theorem}
    A subset of $\RR^n$ is compact if and only if it is closed and bounded.
\end{theorem}

 